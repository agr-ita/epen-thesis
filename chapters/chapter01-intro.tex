{\color{red}[TODO]: Find better titles T-T}

\section{Subjects \& goals // Overview: What is this (epen)thesis about?}

\section{Epenthesis [main]}

\begin{itemize}
\item What is epenthesis?
\item Epenthesis fossils: loanwords \\
  - loanwords + online: phonotactics' mismatch -> epenthesis/deletions
\item Focus on perceptual vowel epenthesis
\item When does epenthesis happen?
  \begin{itemize}
  \item JP:
  \item KR:
  \item BP (but not PP):
  \item ES:
  \item EN:
  \item MCh:
  \end{itemize}
  
\item Neurobiology of epenthesis \\
  - Dehaene2000: Japanese subjects almost never distinguished /CC/ from /CuC/. Native language phonotactics modifies speech perception; vowel epenthesis is by-production of perception processes. Electrophysiological results: responses observed in FR were either not present of shorter + weaker in JP subjects. Fast and automatic coding of the speech input -> compatible with Coarse Coding models of speech perception (but not Segmental or Hierarchical). 

\end{itemize}

\section{Theories of nonnative speech processing [how]}

\begin{itemize}
\item Native speech perception (categorical; development)
\item Best / Kuhl / Flege / Burnham / Liberman\&Mattingley...
\end{itemize}

\section{Input representations of nonnative speech [what]}

\begin{itemize}
\item Representation of the input
  \begin{itemize}
  \item Phonological approach (e.g., cf loanword literature Lovins75, Yip93, Shinohara04) \\
    - Adapted (epenthetic) output corresponds to best match to input with certain order of faithfulness/markedness constraints. Underlying representation (UR) in native language assumed to be close to nonnative form (i.e., abstract wordform $\rightarrow$ ``repair'' by grammar (constraint filter)
    - Feature-based proximity (Paradis \& LaCharité)

  \item Acoustic/phonetic \\
    - PepkDpx2003: phonetically minimal modification \\
    - cf Steriade2001's P-map
  \item Articulatory \\
    - articulation on perception (D'Ausilio, Berent, etc)
  \item Gestures \\
    - ZhaoBerent2018(?): Epenthesis from read stimuli (i.e., no acoustic input) \\
    - BestTyler2007: "PAM posits that perceivers extract invariants about *articulatory gestures* from the speech signal, rather than forming categories from acoustic-phonetic cues"
  \end{itemize}
\end{itemize}

\section{Adding to the mix}  
 
\begin{itemize}
\item Structural information (phonotactics: surface, syllabic) \\
  KabakIdsardi2007: "best working hypothesis is that violations involving syllable structure instead of consonantal contact affect perception, and neither nasalization nor lateralization have any basis in perception."
  \item Phonological rules / constraints
  \item Lexical/phone frequency (probabilistic info) \\
    KabakIds2007: "a phonological influence of L1 phonotactic knowledge, rather than an effect of frequency, plays a primary role in explaining Korean groups’ performance."

  \item Orthography \\
    - Cf interaction of orthography on vowel adaptation in VendelinPepkmp2006
    - Daland2015: ``Strong evidence that Korean listeners relied on the English orthography in selecting the korean vowel adaptation.''; Perceptual Uncertainty Hypothesis: "Orthography is most likely to constrain loanword adaptation in cases when it is not fully determined by perception/phonology alone."
    
\end{itemize}

\section{Processing workflow}
\begin{itemize}
    \item One-step (Dup2011, Wilson?, Duvarsula), vs. Two-step (Berent2007, Monahan2009)
    \item early process (input syst) vs late process (orthography, differences in production ...)
    \end{itemize}
    
\section{Justifying the modelling approach}
\epigraph{... But why make models?}{\textit{Derek Zoolander, probably.}}
Quantitative testing $~$

\section{Plan}

"However, the condition effect observed in Japanese during the second response suggests that some information about the input could be recovered from other processing systems";