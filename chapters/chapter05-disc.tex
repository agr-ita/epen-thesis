%% Summary (how does it answer to the issues mentioned in the introduction? How does it not?) + future avenues

In this thesis, we investigated the mechanisms underlying perceptual vowel epenthesis, by combining experimental and modelling approaches. In particular, we focused on two main questions:
\begin{enumerate}
\item Is perceptual vowel epenthesis a one-step or two-step process?
\item How does a computational implementation of a one-step proposal fare when quantitatively and qualitatively compared to behavioural results? 
\end{enumerate}

In Chapter 2, we investigated the role of acoustics in determining epenthetic vowel quality, as opposed to a phonological process such as flanking vowel copy. We specifically examined how coarticulation cues present in speech items, either naturally or by splicing, modulated the quality of epenthetic vowels perceived by Brazilian Portuguese and Japanese listeners. We found that participants' response patterns could be better explained by variations in coarticulation, with only a small contribution from flanking vowels. We were able to confirm the protagonist role of acoutic detail on epenthetic vowel quality by simulating the behavioural experiments with exemplar-based models of perception. These models only had access to acoustic information and were able to reproduce modulations of epenthetic vowel quality observed in human results. These results were in support of one-step models of nonnative speech perception, in which the acoustic match and sequence match between the nonnative stimulus and the native percept are optimised simultaneously. On the contrary, two-step models in which sequence match is evaluated after an initial categorisation step, are unable to account for acoustic cues modulating epenthetic vowel quality.    \\


In Chapter 3, given the evidence in the previous chapter, we turned to evaluating one-step models specifically. To do so, we recruited tools from the field of speech engineering and automatic speech recognition (ASR). Namely, we built an implementation of a revese inference one-step proposal (\cite{wilson2013}) by using HMM-GMM speech recognizers. These systems are composed of two independent modules: the acoustic model and the language model, which provide the computations necessary to retrieve the acoustic match and sequence match of a speech input and possible parses, respectively. The optimal transcription can be found by combining the two in a one-step optimisation process. We proposed a novel way of testing such ASR models in an identification task analogous to those used when probing perceptual vowel epenthesis in human participants.
Considering the findings from the previous chapter, we used this method to evaluate the predictive power of the acoustic model. Namely, we assessed whether a speech recognizer with a null language model was able to mirror effects of variation in epenthesis. The underlying hypothesis being that the acoustic model is not only necessary, but also sufficient to explain these effects, without contributions of more abstract phonological processes. Our results from using relatively simple ASR models suggested that a phonological component may indeed be necessary to explain all of the effects tested. In parallel, we highlighted the existing limitations of our current models, which could be improved not only at the level of the model itself, but also at the level of the data used to train them.  \\

Aside from improving our model implementation with more state-of-the-art ASR systems, future work should involve a combination of various psycholinguistic paradigms, to ensure that effects observed in both human and model data are not solely due to the specific task used. While we focused on modelling identification tasks (i.e., \textit{n}-forced choice paradigms), it is also possible to evaluate ASR models using non-metalinguistic tasks, such as the ABX discrimination task \cite{schatz2018}. Testing as many combinations of parameters (e.g., acoustic features, model architectures, input data, experimental paradigms, ...), in a search of replicability, is a necessary step towards elucidating the mechanisms underlying speech perception.\\

A modelling approach combined with the availability of behavioural data, such as the how it was presented in this thesis, allows to quantitatively and qualitatively test well-defined theories of nonnative speech perception. Importantly, the same model architecture can be used to study the phenomenon of interest (here: vowel epenthesis) in a crosslinguistic fashion. Not only by cross-referencing to existing behavioural data, but also by allowing to derive new predictions about nonnative speech perception. Moreover, our modelling approach can be easily adapted and extended to fields outside of the field of nonnative speech perception. We encourage future research to combine experimental and modelling approaches in order to evaluated mathematically- and/or algorithmically-defined psycholinguistic theories. 
