%%%%%%%%%%
% [TODO] %
%%%%%%%%%%
% Nothing for now

%%%%%%%%%%%%%%%%%%%%%%
% Chapter mini-intro %
%%%%%%%%%%%%%%%%%%%%%%

%%% Short BG


%%% Research question + alternatives


%%% Plan


%%%%%%%%%%%%%%%%%%%
% General methods % %
%%%%%%%%%%%%%%%%%%%

\section{General methods}
\subsection{ASR tools as models of perception}

\subsection{Dissecting the model}

\subsubsection{Corpora}
\paragraph{CSJ}
\begin{itemize}
\item General description (type of speech, number of speakers)
\item Alignment?
\item Phone set
\item ...
\end{itemize}

\paragraph{KCSS}
\begin{itemize}
\item General description (type of speech, number of speakers)
\item Alignment?
\item Phone set
\item ...
\end{itemize}

\subsubsection{Features}
\begin{itemize}
\item What are MFCCs
\item VTLN and CMVN
\item Pitch
\item LDA?
\item ...
\end{itemize}

\subsubsection{Acoustic model}
\begin{itemize}
\item Monophone model - also, why?
\item Number of gaussians
\item Number of states
\item ...
\end{itemize}
    
\subsubsection{Language models}
\begin{itemize}
\item WFSTs
\item Words \& phones 
\item n-grams
\item ...
\end{itemize}

\subsubsection{Decoding}
\begin{itemize}
\item Lattice generation
\item Acoustic and LM scores 
\item nbest
\item CTM
\item ...
\end{itemize}

\subsection{Assessing native performance}
\subsubsection{Lexicon-based decoding}
\subsubsection{Phonetic-based decoding}

\subsection{Human-model comparisons}
\subsubsection{Simulations}
\subsubsection{Data analysis: ABC easy as 123}
\paragraph{Reasoning}
\paragraph{Model selection}
\paragraph{Parameter tuning}

%%%%%%%%%%%%
% Parlato2 %
%%%%%%%%%%%%

\section{{\color{red}Parlato2 - duration}}
\subsection{Introduction}
\subsection{Methods}
\subsubsection{Stimuli}

\subsubsection{Language models}
In this section, we assess what kind of phonotactics could be used by Japanese listeners when perceiving foreign speech that does not conform to native phonotactics. We test 4 types of language models (LM): 
\begin{enumerate}
    \item A null language model (0P-LM), which implies that listeners base their decoding of consonant clusters on phonetics alone, without using information on phonotactics
    \item A phone-unigram language model (1P-LM), which implies that listeners do not take neighbouring phonemes into consideration when decoding the consonant clusters
    \item An online phone-bigram language model (2PO-LM), which implies that listeners decode the clusters as they hear them (decoding is done from the start of the item)
    \item A retro phone-bigram language model (2PR-LM), which implies that listeners decode the clusters based on the most recent information (decoding is done from the end of the item)
    \item A batch phone-bigram language model (2PB-LM), which implies that listeners decode the item considering the entire structure, with bigrams %use bigram transitional probabilities from the entire cluster, in order to find a global optimal transcription
    \end{enumerate}

    \begin{figure}[htb]
    \centering
%    \includegraphics[width=\linewidth]{}
    \caption{Constrained language model used to test the models. Nodes in the graph represent states, weighted edges represent transitions between states (here: phonemes). Probabilities given by this LM are combined with those from the AM when decoding experimental items.}
    \label{fig:G_fsa}
\end{figure}
\subsection{Results}
\subsection{Discussion}


%%%%%%%%%%%%
% m-/ahpa/ %
%%%%%%%%%%%%

\section{{\color{red}m-/ahpa/}}
\subsection{Introduction}
\subsection{Methods}
\subsection{Results}
\subsection{Discussion}

%%%%%%%%%%%%%%%%%%%%%%%%%%%
% Chapter mini-discussion %
%%%%%%%%%%%%%%%%%%%%%%%%%%%

%%% Summary

%%% Short discussion

%%% Limitations

%%% Conclusions